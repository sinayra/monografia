Nos cursos de computação, a primeira matéria de programação é fundamental para que o aluno possa progredir na área, porém sua aprendizagem não é trivial. Um dos métodos recorridos para auxiliar o ensino, independente da área, é o uso de recursos gráficos. Desta forma, foi proposta uma biblioteca gráfica denominada \playAPC{} para ser aplicada no primeiro semestre da matéria de computação dos cursos de Engenharia da Computação, Engenharia Mecatrônica, Ciência da Computação e Licenciatura em Computação, na disciplina \acrfull{APC} da \acrfull{UnB}. A biblioteca utiliza como recurso gráfico a \acrshort{API} OpenGL e foi feita na linguagem C++. Apesar disso, essas escolhas de tecnologia em nada interferem no processo de aprendizado da matéria \acrshort{APC} que, para alguns professores, é ministrada na linguagem C. Esta monografia descreve o processo desde o levantamento de requisitos para o desenvolvimento da biblioteca até a sua aplicação em aulas de laboratório.

