In computer's relative courses, the first programming subject is essencial for the undergraduate student grow up in the area, but his learning is not trivial. One of the methods to assist the teaching, regardless the area, is the use of graphic resources. It was proposed a graphic libray called \playAPC{} to be applied in the first semester at \acrshort{UnB}, in Computer's Engineering course, Mechatronic Engineering course, Bacharel on Computer Science course and Major in Computer course, in the discipline \acrshort{APC}. The library uses as graphic resources the \acrshort{API} OpenGL and was written in C++ programming language. Even with theses tecnology choises, it will not complicate the learning processing in the discipline, which, for some professors, it is given in C language. This paper describes the survey process requirement from the development of the library to its application in pratical classes.
