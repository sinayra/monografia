

Com este trabalho, no qual foi proposto o desenvolvimento de uma biblioteca gráfica para ser aplicada na aulas de laboratório, observou-se que houve uma grande aceitação tanto por partes dos professores quanto por partes dos alunos. Os professores propuseram exercícios não listados na apostila, além de experimentarem a biblioteca em projetos pessoais que não estão na ementa de \acrshort{APC}, e os alunos utilizaram todos os recursos da biblioteca para resolver os exercícios de forma criativa. Sendo assim, concluímos com êxito o objeto de desenvolver uma biblioteca gráfica que pode ser utilizada por programadores com pouca ou nenhuma experiência em computação gráfica, inclusive de forma didática para os alunos do primeiro semestre da disciplina \acrshort{APC}.

As  principais dificuldades encontradas  por parte
dos alunos que  a experimentaram foi em relação  a sua instalação, uma
vez  que os  mesmos  dificilmente tiveram  contato  com instalaçao  de
bibliotecas  e  \acrshort{API}.  Portanto,  a  necessidade  de  um instalador  é
evidente, especialmente para usuários  Windows e Mac.  Após esse processo de
instalação, a utilização da biblioteca se torna simples, já que é oferecido o
guia de referência das funções disponibilizadas pela \playAPC. Como sugestão de trabalho com a \playAPC{}, uma pesquisa mais detalhada sobre o desempenho de turmas seria interessante, avaliando se a biblioteca aumenta o interesse dos alunos na matéria e, consequentemente, se diminui a evasão no curso.

Por  parte  dos  professores,  a   maior  queixa  foi  em  relação  ao
desempenho, pois programas que necessitam níveis profundos de recursão
ficam  prejudicados por conta de sua arquitetura.  Uma  requisição também por
parte dos  professores envolvidos com  o desenvolvimento da biblioteca  foi a
manipulação de objetos 3D, uma vez que OpenGL foi desenvolvida visando
manipular objetos 3D de forma eficiente. Apesar de ter se iniciado o desenvolvimento de uma biblioteca 3D, a \acrfull{SUGOI}, o projeto não foi concluído. A \acrshort{SUGOI} deveria ser o paralelo da \playAPC{} envolvendo geometrias em 3D, porém a \playAPC{} evoluía em um ritmo mais acelerado do que era possível implementar funcionalidades para a \acrshort{SUGOI}.
Apesar da ausência 3D,
acreditamos que a \playAPC{} possa ajudar no desenvolvimento acadêmico
dos discentes na \textsf{UnB}.