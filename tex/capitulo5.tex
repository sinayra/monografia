
Com intuito de tornar lúdico  o aprendizado de práticas de programação
e  ressaltar a  importância de  matérias do  Departamento de  Física e
Matemática,  como exercícios  simples  de balística  e manipulação  de
formas geométricas,  a \playCB{}  também se torna  uma API  para fazer
pequenas animações ou até mesmo  jogos simples. Entretanto, como um de
seus requisitos  foi legibilidade e compatibilidade, o  uso intenso de
\emph{callbacks} dentro  da API e  sua própria arquitetura  em camadas
não garantem um  bom desempenho dependendo da aplicação  que o usuário
está construindo.   Por exemplo, um jogo utilizando  diretamente a API
OpenGL  terá um melhor  desempenho e  otimização do  que o  mesmo jogo
usando  a \playCB;  a diferença  é que  o primeiro  será feito  por um
estudante  do  quarto  semestre  (preferencialmente além)  enquanto  o
segundo será feito por um estudante do primeiro semestre.